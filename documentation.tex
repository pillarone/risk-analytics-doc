%%
%% Preamble -- sets the document class, loads required packages and defines their defaults and some 'global' constants
%%
\documentclass[11pt,english]{book}
%\usepackage[latin1]{inputenc}% This is useful for catching some non-ascii characters when rich text was pasted into a TeX source file
%\usepackage[T1]{fontenc}% produces poor PDF output!! % but useful for troubleshooting LaTeX Font Warnings from undefined font shapes
\usepackage{amssymb,amsmath,amsfonts}
% load color before hyperref to set link colors in the hyperref options
% for the 'usenames' option, see e.g.:
% http://en.wikibooks.org/wiki/LaTeX/Colors#The_68_standard_colors_known_to_dvips
\usepackage[usenames,dvipsnames]{color}
\definecolor{light-gray}{gray}{0.9}
\definecolor{bgListing}{gray}{0.9}
\definecolor{fgLob}{rgb}{0,0.36,0}
\definecolor{fgCodeComment}{rgb}{0.133,0.545,0.133}
\definecolor{fgCodeLiteral}{rgb}{0.627,0.126,0.941}
% load hyperref before geometry with pdftex & no truedimen (see geometry package doc v4.2 2008-12-21, �10 known problems)
% choose colors for different link types, using names above, either defined explicitly, or from the 'usenames' option
\usepackage[hyperfootnotes=false,%
            colorlinks=true,%
            linkcolor=Blue,%
            anchorcolor=Black,%
            citecolor=MidnightBlue,%
            filecolor=Sepia,%
            menucolor=RoyalBlue,%
            runcolor=BrickRed,%
            urlcolor=Blue%
           ]{hyperref}
\usepackage[margin=2cm,paper=a4paper]{geometry}
\usepackage{graphicx}
\usepackage{listings}
\usepackage{caption}
\usepackage{xspace}
\usepackage{subfig}
%\usepackage{subfigure}
%\usepackage[eng]{babel}
%\usepackage{amsfonts}
\usepackage{framed}
\usepackage{makeidx}
% todo(bgi): if the above isn't working for TeX4HT creation, perhaps try looking at lshort-base.tex & lshort.sty from the source for \cite{Oetiker.2009} in addition to \cite{Gurari.2004} & \cite{Polewczak.2008}
%\usepackage{lastpage}
\usepackage{ifthen}
\usepackage{rotating}

\setlength{\parindent}{0pt}
%todo(bgi): set parindent to 0 also inside footnotes?

% set default values for (code) listings
\lstset{
	numbers=left, numberstyle=\tiny, numbersep=10pt, stepnumber=5,
	backgroundcolor=\color{bgListing},
	tabsize=4,
	rulecolor=,
	language=Java,
	basicstyle=\footnotesize\sffamily,%\scriptsize\footnotesize\small,%\ttfamily\sffamily
	%upquote=true,
	aboveskip={1.5\baselineskip},
	columns=fullflexible,%fixed/flexible/fullflexible/spaceflexible
	extendedchars=true,
	breaklines=true,
	prebreak=\raisebox{0ex}[0ex][0ex]{\ensuremath{\hookleftarrow}},
	frame=single,
	showtabs=false,
	showspaces=false,
	showstringspaces=false,
	identifierstyle=\sffamily,%\ttfamily\sffamily
	keywordstyle=\color[rgb]{0,0,1},
	commentstyle=\color{fgCodeComment},
	stringstyle=\color{fgCodeLiteral},
	%todo:try out \textsf for all listing text!
}

% start indexing
\makeindex

%%
%% Text-style Macros -- defines commands for common styles, strings & indexing
%%
\newcommand{\str}[1]{\textsf{#1}}
\newcommand{\term}[1]{\textbf{#1}}% conceptual term: boldface (todo: when introduced)
\newcommand{\code}[1]{\textnormal{\textsf{#1}}}
\newcommand{\menu}[1]{\textnormal{\textsf{#1}}}
\newcommand{\todo}[2]{\emph{#1}\footnote{TODO: #2}}
\newcommand{\note}[1]{\emph{\textbf{Note:} #1}}
\newcommand{\example}[1]{{\textbf{Example:} #1}}
\newcommand{\caveat}[1]{\textbf{Caveat:} #1}
\newcommand{\filename}[1]{\textsf{#1}}
\newcommand{\squot}[1]{`{#1}'}
\newcommand{\dquot}[1]{``{#1}''}
\newcommand{\tightitemize}[1]{\setlength{\itemsep}{#1}\setlength{\parskip}{0pt}\setlength{\parsep}{0pt}}
\newcommand{\givecredit}[1]{\textit{Source}: #1}
%\newcommand{\note}[1]{\begin{framed}\emph{\textbf{Note:} #1}\end{framed}}
%\newcommand{\caveat}[1]{\begin{framed}\textbf{Caveat:} #1\end{framed}}
%TODO(caveat): put in a yellow box with increased R & L margins and a triangle "caution" icon to the left, inside the box

%%
%% Indexing Macros (proposed convention: indexing commands start with 'ix')
%%
\newcommand{\ixent}[1]{% Organizations, entities
  \textnormal{#1}%
  \index{#1@\textnormal{#1}}%
}
% person
% source: http://latex.silmaril.ie/formattinginformation/chapter9.html#macnest
% e.g.: \person{John Smith} displays `John Smith' but is indexed as `Smith, John'
\newcommand{\ixperson}[1]{%
  #1\reindex #1\sentinel%
}
\def\reindex #1 #2\sentinel{\index{#2, #1}}
%% 
\newcommand{\ixconffile}[1]{%
  \textsf{#1}%
  \index{RiskAnalytics!configuration files!#1@\textsf{#1}}%
}
\newcommand{\ixparmfile}[1]{%
  \filename{#1}%
  \index{RiskAnalytics!parameterization files!#1@\filename{#1}}%
}
\newcommand{\ixpcc}[1]{% Components of a parametrization: [[Dynamic] Composed] Component Containers visible in the GUI parametrization treeview
  \filename{#1}%
  \index{RiskAnalytics!parameterization components!#1@\filename{#1}}%
}
\newcommand{\ixmenu}[1]{%
  \textsf{#1}%
  \index{RiskAnalytics!menu commands!#1@\textsf{#1}}%
}
\newcommand{\ixpkg}[1]{%
  \textsf{#1}%
  \index{RiskAnalytics!packages!#1@\textsf{#1}}%
}
\newcommand{\ixmeth}[1]{%
  \textsf{#1}%
  \index{RiskAnalytics!methods!#1@\textsf{#1}}%
}
\newcommand{\ixclass}[1]{%
  \textsf{#1}%
  \index{RiskAnalytics!classes!#1@\textsf{#1}}%
}
\newcommand{\ixclasstype}[1]{%
  \textsf{#1}%
  \index{RiskAnalytics!static types!#1@\textsf{#1}}%
}
\newcommand{\ixlob}[1]{%
  \textsf{\color{fgLob}{#1}}%
  \index{model parameter examples!lines of business!#1@\textsf{\color{fgLob}{#1}}}%
}



% general indexing macro for any word with a given parent but with no special formatting
% param 1 could be, e.g., 'RiskAnalytics!external libraries!' and param 2 could be 'SSJ'
% or param 1 could be '' and param 2 could be 'Swiss Solveny Test'
\newcommand{\ix}[2]{%
  \index{#1#2@#2}%
  #2% here the text is regurgitated
}
% macro for `see' index entries; note that param 2 uses `relative' path, ignoring parents
\newcommand{\ixsee}[2]{% note: nothing is displayed
  \index{#1|see{#2}}%
}
\ixsee{RiskAnalytics!classes!of objects}{\emph{under} static types}%
\ixsee{RiskAnalytics!models!PODRA}{PODRA~model}%

%%
%% Strings
%%
%% insurance
%\newcommand{\SolvencyII}{Solvency~II{}}% currently unused!
\newcommand{\SST}{SST{}\index{SST|see{Swiss~Solvency~Test}}}
\newcommand{\sst}{\ix{}{Swiss~Solvency~Test}}% to index the a location without displaying any text
%% mathematics
%% numerics
%\newcommand{\FFT}{FFT\index{FFT|see{Fast~Fourier~Transform}}}
%\newcommand{\fft}{\ix{}{Fast~Fourier~Transform}}
%% software
\newcommand{\RiskAnalytics}{\texttt{RiskAnalytics}{}}
\newcommand{\riskanalytics}{\RiskAnalytics}
\newcommand{\RA}{\RiskAnalytics}
\newcommand{\PillarOne}{\texttt{PillarOne}{}}
\newcommand{\pillarone}{\PillarOne}
\newcommand{\PO}{\PillarOne}
\newcommand{\PORA}{\PillarOne~\RA}
\newcommand{\PODRA}{Podra{}\index{PODRA~model}}
% Java: display, reference & index
% format, link & index the word Java when it denotes the computer language
\newcommand{\Java}{\href{Java}{http://java.sun.com/}\index{Java}}
% format & index the word Groovy when it denotes the computer scripting language
\newcommand{\Groovy}{Groovy{}% no special formatting yet for 'Groovy' (aside from name capitalization)
 \index{Groovy}}
\newcommand{\GroovyClosure}[1]{closure% no special formatting yet for 'closure'
 \ifthenelse{\equal{#1}{true}}{%
 \index{Groovy!closure|textbf}}{%
 \index{Groovy!closure}}}
%% files (commonly mentioned)
\newcommand{\ConfigGroovy}{\ixconffile{Config.groovy}}
\newcommand{\AppProps}{\ixconffile{application.properties}}
%% entities/organizations to display, index and hyperlink (in the index)
\newcommand{\MR}{Munich~Re{}\index{MunichRe@\textbf{\href{http://www.munichre.com}{Munich~Re}}}}
\newcommand{\IC}{Intuitive~Collaboration{}\index{Intuitive~Collaboration@\textbf{\href{http://www.intuitive-collaboration.com}{Intuitive~Collaboration}}}}
\newcommand{\Canoo}{Canoo{}\index{Canoo@\textbf{\href{http://www.canoo.com}{Canoo}}}}
\newcommand{\FinMa}{FinMa{}\index{FinMa@\textbf{\href{http://www.finma.ch/}{FinMa}}}}
\newcommand{\FINMA}{FINMA{}\index{FinMa@\textbf{\href{http://www.finma.ch/}{FinMa}}}}

\newcommand{\ie}{i.\,e.\@\xspace}
\newcommand{\eg}{e.\,g.\@\xspace}
\newcommand{\Eg}{E.\,g.\@\xspace}
\newcommand{\etc}{etc.\@\xspace}
\newcommand{\ff}{ff.\@\xspace}
\newcommand{\cf}{cf.\@\xspace}

\makeindex

\begin{document}

%\tableofcontents

\chapter*{Documentation Guidelines for \PO}
\today

\section{Prerequisites}

You need a \TeX{}/\LaTeX{} distribution, and an editor for \TeX{}. For Windows we're currenly using:
\begin{itemize}
  \item \LaTeX{} distribution: \href{http://miktex.org/}{MiKTeX} (Note: MiKTeX automatically downoads
	  any required packages such as, e.g., \href{http://www.cse.ohio-state.edu/~gurari/TeX4ht/}{TeX4HT}
	  for producing HTML+image \& XML+MathML output formats)
  \item Editor: \href{http://www.texniccenter.org/}{TeXnicCenter}
\end{itemize}
although there are many other possibilities (see, e.g., section
\href{http://www.csun.edu/~hcmth008/mathml/acc_tutorial.html#x1-120006.3}{6.3} of \href{http://www.csun.edu/~hcmth008/mathml/acc_tutorial.pdf}{Polewczak's {TeX4HT} Tutorial}).

\section{Practical Documentation Rules}

We are using a few macros for some standard types of text such as Java/Groovy code, terms, notes, caveats, etc. See Section \ref{sec:ourmac} below.

We agree to observe the following rules:
\begin{itemize}
  \item Have one master \TeX{} file that contains all the inclusion commands; only use \code{{\textbackslash}input} for including other \TeX{} sources, and only from the master file.
  \item We have set up some macros for common use. Please use these whenever appropriate. We will expand or refine these as needed. If you find yourself regularly using your own LaTeX macros, let us know!
  \item \code{{\textbackslash}label} names must be `typed', eg: sec, subsec, subsubsec, fig, lst for, respectively, sections, subsections, sub-subsections, figures and listings. For example, a section called 'documentation' should be \TeX ed as: \code{{\textbackslash}label\{sec:documentation\}}.
  \item Say in a comment why you are including each package.
  \item \ldots
\end{itemize}
%The intention is to have some sensible TeX coding conventions which also allow some flexibility --- for example, the production of PDF, HTML or XML output formats with minimal fuss.

\section{Our Setup}

TeXnicCenter has a project file (\code{.tcp}) to specify the main \TeX{} file when a project contains more than one. Sample \code{tcp} files are included in version control for the \code{RiskAnalyticsMaster.tex} file (combined \RA{} Guides) and the \code{documentation.tex} file (Documentor's Guide). Both are currently configured for PDF output, but may be changed for HTML or XML output (using TeX4HT) if one re-organizes the heirarchical structure slightly (by splitting chapters, for example)\todo{revise \TeX{} sources to re-harmonize with TeX4HT/XML output}. This is because \code{book} \code{documentstyle}s allow a level 0 \code{{\textbackslash}chapter}, whereas \code{article}s start with level 1 \code{{\textbackslash}section}s.

\section{Document Structure Conventions}

We have split the \TeX{} preamble and macro sections into separate files. A typical `master' \TeX{} file would then look like Listing \ref{lst:structbook}.
So typically you would write your contribution as a \code{{\textbackslash}section} in an existing or new \TeX{} file and it would be included in the `master' \TeX{} file with an \code{{\textbackslash}input} statement.

\begin{framed}
\begin{lstlisting}[label=lst:structbook, caption={Document Structure Sample: book documentstyle}, language=TeX]
%%
%% Preamble -- sets the document class, loads required packages and defines their defaults and some 'global' constants
%%
\documentclass[11pt,english]{book}
%\usepackage[latin1]{inputenc}% This is useful for catching some non-ascii characters when rich text was pasted into a TeX source file
%\usepackage[T1]{fontenc}% produces poor PDF output!! % but useful for troubleshooting LaTeX Font Warnings from undefined font shapes
\usepackage{amssymb,amsmath,amsfonts}
% load color before hyperref to set link colors in the hyperref options
% for the 'usenames' option, see e.g.:
% http://en.wikibooks.org/wiki/LaTeX/Colors#The_68_standard_colors_known_to_dvips
\usepackage[usenames,dvipsnames]{color}
\definecolor{light-gray}{gray}{0.9}
\definecolor{bgListing}{gray}{0.9}
\definecolor{fgLob}{rgb}{0,0.36,0}
\definecolor{fgCodeComment}{rgb}{0.133,0.545,0.133}
\definecolor{fgCodeLiteral}{rgb}{0.627,0.126,0.941}
% load hyperref before geometry with pdftex & no truedimen (see geometry package doc v4.2 2008-12-21, �10 known problems)
% choose colors for different link types, using names above, either defined explicitly, or from the 'usenames' option
\usepackage[hyperfootnotes=false,%
            colorlinks=true,%
            linkcolor=Blue,%
            anchorcolor=Black,%
            citecolor=MidnightBlue,%
            filecolor=Sepia,%
            menucolor=RoyalBlue,%
            runcolor=BrickRed,%
            urlcolor=Blue%
           ]{hyperref}
\usepackage[margin=2cm,paper=a4paper]{geometry}
\usepackage{graphicx}
\usepackage{listings}
\usepackage{caption}
\usepackage{xspace}
\usepackage{subfig}
%\usepackage{subfigure}
%\usepackage[eng]{babel}
%\usepackage{amsfonts}
\usepackage{framed}
\usepackage{makeidx}
% todo(bgi): if the above isn't working for TeX4HT creation, perhaps try looking at lshort-base.tex & lshort.sty from the source for \cite{Oetiker.2009} in addition to \cite{Gurari.2004} & \cite{Polewczak.2008}
%\usepackage{lastpage}
\usepackage{ifthen}
\usepackage{rotating}

\setlength{\parindent}{0pt}
%todo(bgi): set parindent to 0 also inside footnotes?

% set default values for (code) listings
\lstset{
	numbers=left, numberstyle=\tiny, numbersep=10pt, stepnumber=5,
	backgroundcolor=\color{bgListing},
	tabsize=4,
	rulecolor=,
	language=Java,
	basicstyle=\footnotesize\sffamily,%\scriptsize\footnotesize\small,%\ttfamily\sffamily
	%upquote=true,
	aboveskip={1.5\baselineskip},
	columns=fullflexible,%fixed/flexible/fullflexible/spaceflexible
	extendedchars=true,
	breaklines=true,
	prebreak=\raisebox{0ex}[0ex][0ex]{\ensuremath{\hookleftarrow}},
	frame=single,
	showtabs=false,
	showspaces=false,
	showstringspaces=false,
	identifierstyle=\sffamily,%\ttfamily\sffamily
	keywordstyle=\color[rgb]{0,0,1},
	commentstyle=\color{fgCodeComment},
	stringstyle=\color{fgCodeLiteral},
	%todo:try out \textsf for all listing text!
}

% start indexing
\makeindex

%%
%% Text-style Macros -- defines commands for common styles, strings & indexing
%%
\newcommand{\str}[1]{\textsf{#1}}
\newcommand{\term}[1]{\textbf{#1}}% conceptual term: boldface (todo: when introduced)
\newcommand{\code}[1]{\textnormal{\textsf{#1}}}
\newcommand{\menu}[1]{\textnormal{\textsf{#1}}}
\newcommand{\todo}[2]{\emph{#1}\footnote{TODO: #2}}
\newcommand{\note}[1]{\emph{\textbf{Note:} #1}}
\newcommand{\example}[1]{{\textbf{Example:} #1}}
\newcommand{\caveat}[1]{\textbf{Caveat:} #1}
\newcommand{\filename}[1]{\textsf{#1}}
\newcommand{\squot}[1]{`{#1}'}
\newcommand{\dquot}[1]{``{#1}''}
\newcommand{\tightitemize}[1]{\setlength{\itemsep}{#1}\setlength{\parskip}{0pt}\setlength{\parsep}{0pt}}
\newcommand{\givecredit}[1]{\textit{Source}: #1}
%\newcommand{\note}[1]{\begin{framed}\emph{\textbf{Note:} #1}\end{framed}}
%\newcommand{\caveat}[1]{\begin{framed}\textbf{Caveat:} #1\end{framed}}
%TODO(caveat): put in a yellow box with increased R & L margins and a triangle "caution" icon to the left, inside the box

%%
%% Indexing Macros (proposed convention: indexing commands start with 'ix')
%%
\newcommand{\ixent}[1]{% Organizations, entities
  \textnormal{#1}%
  \index{#1@\textnormal{#1}}%
}
% person
% source: http://latex.silmaril.ie/formattinginformation/chapter9.html#macnest
% e.g.: \person{John Smith} displays `John Smith' but is indexed as `Smith, John'
\newcommand{\ixperson}[1]{%
  #1\reindex #1\sentinel%
}
\def\reindex #1 #2\sentinel{\index{#2, #1}}
%% 
\newcommand{\ixconffile}[1]{%
  \textsf{#1}%
  \index{RiskAnalytics!configuration files!#1@\textsf{#1}}%
}
\newcommand{\ixparmfile}[1]{%
  \filename{#1}%
  \index{RiskAnalytics!parameterization files!#1@\filename{#1}}%
}
\newcommand{\ixpcc}[1]{% Components of a parametrization: [[Dynamic] Composed] Component Containers visible in the GUI parametrization treeview
  \filename{#1}%
  \index{RiskAnalytics!parameterization components!#1@\filename{#1}}%
}
\newcommand{\ixmenu}[1]{%
  \textsf{#1}%
  \index{RiskAnalytics!menu commands!#1@\textsf{#1}}%
}
\newcommand{\ixpkg}[1]{%
  \textsf{#1}%
  \index{RiskAnalytics!packages!#1@\textsf{#1}}%
}
\newcommand{\ixmeth}[1]{%
  \textsf{#1}%
  \index{RiskAnalytics!methods!#1@\textsf{#1}}%
}
\newcommand{\ixclass}[1]{%
  \textsf{#1}%
  \index{RiskAnalytics!classes!#1@\textsf{#1}}%
}
\newcommand{\ixclasstype}[1]{%
  \textsf{#1}%
  \index{RiskAnalytics!static types!#1@\textsf{#1}}%
}
\newcommand{\ixlob}[1]{%
  \textsf{\color{fgLob}{#1}}%
  \index{model parameter examples!lines of business!#1@\textsf{\color{fgLob}{#1}}}%
}



% general indexing macro for any word with a given parent but with no special formatting
% param 1 could be, e.g., 'RiskAnalytics!external libraries!' and param 2 could be 'SSJ'
% or param 1 could be '' and param 2 could be 'Swiss Solveny Test'
\newcommand{\ix}[2]{%
  \index{#1#2@#2}%
  #2% here the text is regurgitated
}
% macro for `see' index entries; note that param 2 uses `relative' path, ignoring parents
\newcommand{\ixsee}[2]{% note: nothing is displayed
  \index{#1|see{#2}}%
}
\ixsee{RiskAnalytics!classes!of objects}{\emph{under} static types}%
\ixsee{RiskAnalytics!models!PODRA}{PODRA~model}%

%%
%% Strings
%%
%% insurance
%\newcommand{\SolvencyII}{Solvency~II{}}% currently unused!
\newcommand{\SST}{SST{}\index{SST|see{Swiss~Solvency~Test}}}
\newcommand{\sst}{\ix{}{Swiss~Solvency~Test}}% to index the a location without displaying any text
%% mathematics
%% numerics
%\newcommand{\FFT}{FFT\index{FFT|see{Fast~Fourier~Transform}}}
%\newcommand{\fft}{\ix{}{Fast~Fourier~Transform}}
%% software
\newcommand{\RiskAnalytics}{\texttt{RiskAnalytics}{}}
\newcommand{\riskanalytics}{\RiskAnalytics}
\newcommand{\RA}{\RiskAnalytics}
\newcommand{\PillarOne}{\texttt{PillarOne}{}}
\newcommand{\pillarone}{\PillarOne}
\newcommand{\PO}{\PillarOne}
\newcommand{\PORA}{\PillarOne~\RA}
\newcommand{\PODRA}{Podra{}\index{PODRA~model}}
% Java: display, reference & index
% format, link & index the word Java when it denotes the computer language
\newcommand{\Java}{\href{Java}{http://java.sun.com/}\index{Java}}
% format & index the word Groovy when it denotes the computer scripting language
\newcommand{\Groovy}{Groovy{}% no special formatting yet for 'Groovy' (aside from name capitalization)
 \index{Groovy}}
\newcommand{\GroovyClosure}[1]{closure% no special formatting yet for 'closure'
 \ifthenelse{\equal{#1}{true}}{%
 \index{Groovy!closure|textbf}}{%
 \index{Groovy!closure}}}
%% files (commonly mentioned)
\newcommand{\ConfigGroovy}{\ixconffile{Config.groovy}}
\newcommand{\AppProps}{\ixconffile{application.properties}}
%% entities/organizations to display, index and hyperlink (in the index)
\newcommand{\MR}{Munich~Re{}\index{MunichRe@\textbf{\href{http://www.munichre.com}{Munich~Re}}}}
\newcommand{\IC}{Intuitive~Collaboration{}\index{Intuitive~Collaboration@\textbf{\href{http://www.intuitive-collaboration.com}{Intuitive~Collaboration}}}}
\newcommand{\Canoo}{Canoo{}\index{Canoo@\textbf{\href{http://www.canoo.com}{Canoo}}}}
\newcommand{\FinMa}{FinMa{}\index{FinMa@\textbf{\href{http://www.finma.ch/}{FinMa}}}}
\newcommand{\FINMA}{FINMA{}\index{FinMa@\textbf{\href{http://www.finma.ch/}{FinMa}}}}

\newcommand{\ie}{i.\,e.\@\xspace}
\newcommand{\eg}{e.\,g.\@\xspace}
\newcommand{\Eg}{E.\,g.\@\xspace}
\newcommand{\etc}{etc.\@\xspace}
\newcommand{\ff}{ff.\@\xspace}
\newcommand{\cf}{cf.\@\xspace}

\makeindex
\begin{document}

%%
%% Top Matter
%%

\begin{titlepage}
%\begin{minipage}
%
% Include the logo image.
% JPEG images work in both pdflatex and latex, but need a bounding box (bb) specification in the latter.
\title{\includegraphics[width=9cm]{images/pillarone-logo-simple-small.jpg}\\[2cm]
%\title{\includegraphics[width=11cm]{images/pillarone-logo-simple-small.jpg}\\[2cm]
% the above options work well for PDF (pdflatex), but appears rather large in HTML w/o bb specification;
% something like the below works better for tex4ht HTML (htlatex) or XML (mzlatex) production:
%\title{\includegraphics[width=11cm, bb=0 0 693 282]{images/pillarone-logo-simple-small.jpg}\\[2cm]
%TODO: improve the bb options for tex4ht (HTML/XML) targets.
%
%%\label{logo:pillarone}
%%\comment{Risk Management meets Open Source}
%%\begin{flushright}{\textsc{Risk Management meets Open Source}}\end{flushright}\\[0cm]
%\textcolor{light-gray}{\rule{\linewidth}{1pt}}\\[1cm]
{\Huge \textsf{RiskAnalytics 1.0}}\\[1cm]}
\author{}%\PO core team
\date{29 February 2012}%\today
\maketitle

\begin{flushleft}
{\small{
{\bfseries Version}: 1.1.\pageref{LastPage}\\[1cm]

\emph{Copies of this document may be made for your own use and for distribution to others, provided that you do not charge any fee for such copies and further provided that each copy contains this Copyright Notice, whether distributed in print or electronically.}\\[1cm]

The \RA{} source code and the example code presented in this book are available % in the code repository
at\index{RiskAnalytics!source code|textbf}\\
\url{https://svn.intuitive-collaboration.com/RiskAnalytics/trunk/RiskAnalyticsPC}\\[1cm]

This document has been written by various \PO{} core team members:
\begin{itemize}\tightitemize{0pt}
  \item Jon Bardola, FS-Consulta GmbH
	\item J\"org Dittrich, Munich Re
	\item Benjamin Ginsberg, Intuitive Collaboration AG
	\item Fouad Jaada, Intuitive Collaboration AG
	\item Stephan Hartmann, Munich Re
%	\item Dierk K{\"o}nig% has he written content?
	\item Stefan Kunz, Intuitive Collaboration AG
	\item Markus Meier, Intuitive Collaboration AG
	\item	Martin Melchior, Fachhochschule Nordwestschweiz
	\item Michael Spahn, Intuitive Collaboration AG
	\item Markus Stricker, Intuitive Collaboration AG
	\item Jessika Walter, Intuitive Collaboration AG
%	\item \ixperson{{Johann Wolfgang} {von Goethe}}%example: displays correctly but is sorted in index under `v'
\end{itemize}
\vspace*{1cm}

The \PO{} project was initiated and sponsered by \MR.
}}

\vspace*{1cm}

For further details please get in touch: \href{mailto:contact@pillarone.org}{contact@pillarone.org}

\end{flushleft}
\end{titlepage}



%\end{minipage}

%\pagebreak

\tableofcontents

\chapter{chapter 1 title}
\chapter{Introduction}
\label{chap:refguide-intro} 

Solvency~II can be seen as a driver for \RA, but it is certainly not the only one. In general, it is the trend towards embedding the quantitative output of actuarial and risk management models in operative processes. This requires more than just correct calculations. Issues that are becoming more important are: `where did the input data come from?', `who owns the data and who has the right to edit the data -- or to sign it off?', `how do we get the input data in an operationally safe way into the modeling tools?', `how do we get the output data out of the modeling tool for reporting?', 'is the used version documented?', `can an auditor or a regulator review the complete solution efficiently?', etc. In short, we forsee that actuarial and risk management applications will have to reach the same level of reliability, integration and security as financial applications.\footnote{Not long ago, the financial statements of a group of companies was consolidated using spreadsheets and copy-pasting information from back and forth. While many risk management applications still rely on this approach, nobody could imagine not using professional consolidation software these days.}

Most actuarial modeling tools cover only the quantitative aspects of actuarial models. \RA{} strives to provide a sound base for a more complete risk management or Solvency~II solution. The quantitative aspects of the Solvency~II framework -- `Pillar~One' -- gave the software suite its name. But {\PO.\RA}, or in short \RA, is more than just an actuarial calculation engine. Auditability, security and process support, which are necessary for `Pillar~Two' in the Solvency~II framework are also part of \RA. `Pillar~Three' of the Solvency~II framework involves disclosure and transparency, which is related to reporting standards.  The calculation engine of \RA{} can provide the data for internal as well as external reporting, using industry-standard interfaces for professional reporting.

\PO{} was initiated at the end of 2007 and sponsored by \href{http://www.munichre.com}{\MR{}}. Apart from \MR, \href{http://www.Intuitive-Collaboration.com}{\IC{}} and \href{http://www.canoo.com}{\Canoo{}} provided major resources for the developement of the software.

In a nutshell, \PO{} can be characterized by
\begin{itemize}
  \item {\bf Transparency} is a major value in risk management. \PO{} provides the ultimate transparency: all methods and implementations are licensed under an open source licence (\href{http://www.gnu.org/licenses/gpl-3.0.html}{GPL v3}) which guarantees that anybody can get unrestricted access to the descriptions of the used methods and their implementations. Anybody is allowed to change or extend the implementation. The only thing which the GPL license forbids is to sell the whole or parts of the source code or to wrap it in a product with a commercial license. 
  \item {\bf IT Standards} are virtually nonexistent in most actuarial tools. \PO{} is a welcome exception since it is built on top of broadly accepted Java enterprise software components for database handling, client-server communication, security, etc. In short, \RA{} is a true enterprise application.
  \item {\bf Flexibility} is required for a platform to cover a broad spectrum of applications. \RA{} makes no assumptions about the time resolution of models, the level of detail to be modeled, or which output data will be collected. As a result, \RA{} can be used for a broad spectrum of quantitative insurance models, including risk capital or Solvency~II models, reinsurance optimization, portfolio or deal valuations, profit testing \ldots to mention a few.
\end{itemize}

Beyond the non-functional, or architectural, requirements mentioned above, \RA{} offers a number of cool usability (and other) features. We mention a few below, with links to further elaborations for those with whetted appetites.

% In this list, we refer to some special features of RiskAnalytics. This needs to be done in two ways: firstly, listing cool stuff here and putting a reference to an example; second by highlighting the example with the command \ any ideas???
% One idea for highlighting these examples would be to place them in thematized boxes, as is often found in IT and educational books. A possible macro example (using the minipage environment -- which, beware, does not split across pages) is given at http://stackoverflow.com/questions/1901213/open-source-latex-environment-for-educational-books.
\index{RiskAnalytics!features}
\begin{itemize}
	\item \label{feature:Compare}
		\term{Compare}: The user can simultaneously textually compare two or more simulation 
		results\footnote{\ixmenu{compare~simulation~results} by first left-clicking them 
		while pressing the Ctrl key to select them, next right-clicking anywhere in the shaded 
		selection to invoke the context menu, and, finally, clicking on the compare option} and/or 
		their parametrizations\footnote{by clicking on the \ixmenu{compare~parametrizations} 
		option from within the result comparison}.
		
		\term{Compare Graphics}: The user can compare results graphically within a given result 
		set\footnote{\ixmenu{compare (claims) distributions} is invoked in a similar fasion}. Smoothing algorithms are provided.
		
	\item \label{feature:Clipboard}
		\term{Seamless clipboard integration}: Clicking on the top left cell of a result 
		section selects all of its data, which can then be copied to the clipboard and pasted 
		to a spreadsheet application with standard menu commands or keyboard shortcuts\footnote{
		\eg, right-clicking for the context menu, using the Edit menu, or using the keyboard 
		shortcuts Ctrl-c and Ctrl-v directly to copy and paste, respectively}.
	\item \label{feature:Dockable}
		\term{Dockable tabs}: Within the application window, multiple tabs can be open 
		simultaneously, but only one tab is active at a time. Dockable tabs allow each tab 
		to be undocked into its own window, or subsequently returned into the main application 
		window, thus enabling the user to view and interact with multiple aspects of the 
		modeling domain in parallel -- for example, to compare or copy data side-by-side; or to open the comments-window separately on the screen wherever you like.
	\item \label{feature:Validation}
		\term{Validators}: Custom `callback' functions implementing buisiness-specific `rules'
		can be written and easily pushed/deployed to adhere to enterprise policies or to 
		enforce data integrity at the time of entry.
	\item \term{Comments}
	\item \term{Multi Company Model (MCM)}
	\item \term{Batch runs}
	\item \term{Views}
	\item \term{}
	\item \term{}
	\item \term{}
	
\end{itemize}
     % => \section{Introduction} ...
\chapter{Getting Started}
\label{chap:refguide-getstart}

Following the instructions in this chapter gets you from a to z within a short time. But only from little a to little z -- the remaining chapters cover the A to Z for \RA. To stick with the illustration, \RA{} knows many different font types, even more than openly available. In this manual, however, we are able to cover only a few. Please get in touch for more details about more fonts, i.e.~how to use \RA{} in a variety of environments such as solvency and pricing, life and non-life, direct and reinsurance, one- or multi-period models:  \href{mailto:contact@pillarone.org}{contact@pillarone.org}

\RA{} comes in two flavors: A stand-alone version which can be installed on a laptop or PC and a server version which is required for an enterprise solution. Since \PO{} is based on Java technology, it runs on Windows, Unix/linux and MacOS X. In the client-server version, the client can run a different OS than the server. 

Laptops have become fairly powerful, but their disks are significantly slower than those on servers. Hence, we recommend the standalone version for evaluation, for development and for anybody who works more or less alone on a modelling project or has no access to a server infrastructure, \eg~consultants on the road.

If you just want to have a first glimps at \RA, then you can also play with the client-server version on our test environment on \href{http://pillarone.org/pillarone/try-it-online}{pillarone.org/pillarone/try-it-online}.\footnote{Be aware that this is a test environment; we may restart the server, clean the database and restrict the number of users or the number of iterations per user.}

\section{Installing the standalone version}
\label{sec:refguide-install-standalone}

The standalone version is suitable to be installed on a laptop or desktop computer with at least 1 GB of RAM and several GB of harddisk space -- needed for handling the results of your simulations properly including all details as per your specification. It comes in an absolutely self-contained installer and, apart from the typical client-server components, it is identical to the client-server version. 

The following steps describe how a standalone version can be installed. 

\begin{enumerate}
	\item Download the latest version from \href{http://pillarone.org/products/modelling}{pillarone.org}.\\ 
		\note{If you have an earlier version of \RA{} installed, 
			you will need to update (uninstall and reinstall) some plugins\ldots If, however, you will no longer need the parameters and results from the earlier installation, you might want to simply and completely uninstall the earlier installation and rightafter install the new version.}
	\item Install
\end{enumerate}

\subsection{Database environments}

In the standard setup, \PO \RA{}comes with an already up and running database. No customization needed. Just use it. 

However, if you want to use another database, the following gives a first introduction: By default there is support for two different embedded databases: \term{mysql} (recommended) and \term{derby}. Which one will be used depends on the script used to start risk analytics.
These databases are started and stopped together with RiskAnalytics, which means that it is not possible to access the database externally when the application is closed.

If required it is also possible to run the application with an installed MySQL database (version 5.1 or newer is required).
The database must be accessible at \texttt{localhost:3306} (3306 being the standard port). A database called \texttt{p1rat} must be created as well as user with name and password \texttt{p1rat}
with the necessary privileges for the before mentioned table.

The database and the user can be setup with the following commands:

\code{
create database p1rat;\\
create user 'p1rat'@'localhost' identified by 'p1rat';\\
grant all on table p1rat.* to 'p1rat'@'localhost';\\
grant file on *.* to 'p1rat'@'localhost';
}

This would be sufficient to use \RA{} with a standalone mysql, however for acceptable performance a script which will setup partitioning and indices should be executed.
It can be downloaded at https://svn.intuitive-collaboration.com/RiskAnalytics/trunk/
\\% linebrake needed
RiskAnalytics/src/java/mysql.sql

The database is now ready for Risk Analytics. To use this database with Risk Analytics it is necessary to edit the start script RunRiskAnalyticsMySql.cmd
and replace \texttt{-Dgrails.env=mysqlembedded} with \texttt{-Dgrails.env=mysql}

\subsection{Reset the database}

To discard all changed data including results just remove the database directory in the pillarone temporary directory, which is \texttt{\~{}/.pillarone} by default, but can be changed during the installation process.
If not using the embedded mysql, but an installed one, just re-run the script used to initialize the database.

\subsection{Accessing results directly} 

If you want to access the results directly they are saved in the table \texttt{single\_value\_result}. Only mysql can be accessed from outside the application (because derby runs in the same JVM).
Keep in mind that when you want to access the results from the embedded mysql database (which runs on port 3307), \RA{} must be running. 

\section{Installing the server version}
\label{sec:refguide-install-server}

This will give you a much more powerful set-up, but we recommend that you do this only if you have ample experience in dealing with server based installations or if you have a test server at your disposition.

Prerequisits: 
\begin{itemize}
	\item A servlet container, \eg~Tomcat. No need for a fully fledged middleware.
	\item A database, \eg~mysql. Note that you may have to edit the datasource 
		information in the file \ixconffile{DataSource.groovy}%
		\footnote{Groovy\index{Groovy|textbf} 
			is an open-source scripting language based on Java.
			See their \href{http://groovy.codehaus.org/}{homepage} 
			or \cite{GroovyIA07} for more information.}.
\end{itemize}


\section{Your first simulation}
\label{sec:firstSimulation}

Start \RA{} by either opening the sandbox model from our server or the already installed version (as explained above). The releases include some demo models and we will use one of them now to verify that the installation is properly working. In the left pane, expand the \PODRA{} model and you should see the three subitems: \menu{Parameters}, \menu{Result templates} and \menu{Results}. 

\todo{Screenshot}{of the application as it opens}

Open the parameters section, right-click on a parameterization that you want to base your simulation on, and then select \menu{run simulation}. This opens the simulation tab and sets the parametrization from which you launched this view. You now want either to keep the suggested result template or make your own choice. For the first test run, entering a value between $100$ and $1000$ in the number of iterations will be an appropriate choice before clicking on \menu{Run}.

\todo{Screenshot}{of the simulation page}

Now lets have a look at the result section. Open the result of the most recent simulation. There are many ways to look at the data: tables, graphics, comparison of results. You will find out more on your personal excursion through your first simulation.

\todo{Screenshot}{result analysis context menu, one or two examples}

\section{A mini result analysis: MRA}
\label{sec:miniResultAnaylsis}

\todo{If you want to have a look at the tiniest -- still meaningful -- model possible, we got something for you: \term{MRA}. Consisting of hard-wired elements such as: one underwriting segment, one claim generator for attritional claims and one proportional reinsurance contract.}{We need such a model and description} 

bls bls blup

\chapter{Concepts}
\label{chap:refguide-concepts}

In order to understand the concepts of \PO{} \RA,
it is necessary to get comfortable with the words that we use
to denote various aspects in their context, \ie

\begin{itemize}
  \item \term{Model} contains all the risk drivers, applying them in a specific order by using a data driven workflow
  \item \term{Components} are the building blocks of a model
  \item \term{Packets} contain the information sent from component to component and persisted as results
  \item \term{Wiring} describes the relations among the components of a model
\end{itemize}

as introduced in Section~\ref{sec:riskmodel} and

\begin{itemize}
  \item \term{Simulation}: executing a model with a specific parameterization collecting results as described in a result template
  \item \term{Iteration}: one possible realization in a stochastic context, also referred to as an `iteration run'; one specific, actual realization in a deterministic context
  \item \term{Period}: projection step; \ie~a calender based time interval, such as a business year
  \item \term{Results}: Resultanalysis, graphically and numericalle by different graphics and \term{views}
\end{itemize}

as introduced in Section~\ref{sec:simulation}.

Before entering into more details, let us give some definitions:

\begin{itemize}
  \item \term{Persistence}
  \item \term{Parameterization}
  \item \term{Result}
  \item \term{Business Logic}
  \item \term{Collection Mode}
  \item \term{}
  \item \term{}
\end{itemize}


\section{Risk Model}
\label{sec:riskmodel}
There are three different kinds of models:
\begin{itemize}
  \item deterministic models especially designed for life modelling
  \item stochastic models with no time concept
  \item stochastic models with a time concept
\end{itemize}

Examples are the models \term{Capital Eagle} or \term{\PODRA{}}.

\subsection{Model}
\subsection{Components} 
\subsection{Packets} 

\subsection{Wiring} 

\section{Simulation}
\label{sec:simulation}

A \term{simulation} runs a \hyperref[sec:riskmodel]{\term{risk model}} a given number of times.
Each of the single executions -- all together makeing up a simulation -- is an \term{iteration}.
In other words, a simulation starts a number of iterations and works on the collected results.
Every iteration runs the model for a given number of \term{periods}.

Periods can most easily be thought of as business years (but they can be any regular, calendar-based time interval).
Each period may have a different set of parameters.
Periods have a natural sequence, and output from one period can be transferred as input to the next period. The number and length of periods is defined in the parameterization. If all periods are of equal length this may be defined in the model. \todo{different lengt?}{; if they are of different length ... }

Schematically, for a simulation with $m$ iterations and a parameterization defining $n$ periods:

\begin{tabular}{|cccccc|}
\hline
&&&&&\\
\multicolumn{6}{|l|}{simulation}\\
&&&&&\\
\hspace{1cm}
 &  &  period $1$ &  period $2$ & \ldots    & period $n$ \\
 & iteration $1$  &  \ldots     &  \ldots     & \ldots    & \ldots     \\
 &   $\vdots$     &  $\vdots$   &  $s_{ip}$   & $\vdots$  & $\vdots$   \\
 & iteration $m$  &  \ldots     &  \ldots     & \ldots    & \ldots     \\
&&&&&\\
\hline
\end{tabular}

\subsection*{Simulation}

In order to execute a simulation, the user has to select a risk model, a parameterization and a result template.

\subsection*{Iteration}

\todo{screen}{shot}

\todo{deterministic models}{how to use them???}
\todo{stochastic models}{how to use them???}

\begin{itemize}
  \item The number of iterations required depends upon the statistical key figures to be evaluated.
        Generally speaking, more iterations are required for distributions with fatter tails, and/or evalutaions of higher/lower quantiles.
  \item A simulation of a DeterministicModel is called \emph{calculation}. It contains one iteration only. It's results are deterministic and not stochastic.
\end{itemize}

\subsection*{Period}

For each period, every component executes the following methods exactly once, after all information from its prerequisite components is available:
\begin{itemize}
  \item \code{doCalculation()} evaluates incoming information and parameters and prepares outgoing information
  \item \code{publishResults()} sends the produced information to following components or collectors
  \item \code{reset()} prepares the component for the next period
\end{itemize}

\subsection*{Results}

The result is a set of {simulation, iteration, period, path, field, value}.

\section{Component}
\label{sec:component}

Components are the \emph{building blocks} of models. Most of the business logic is kept in components. Components have typed
interfaces which receive and send \emph{packets} from and to other components and parameters.

Thinking of a model as a \emph{directed graph}, components are the \emph{nodes} in such a graph. An \emph{edge} links a sender
interface to a receiver interface of another component.

\subsection*{Advice}
\begin{itemize}
  \item Try to keep components simple; a component should do one thing, not many things. This makes it easier to test it and also increases the chances that it can be re-used. Better buiding two components running after one other than only one.
  \item Before starting to implement any component, draw the directed graph with pencil and paper. Specify where specific information is required and produced. Once the directed graph can be mentally traversed, it is time to start coding.
\end{itemize}

\subsection*{Concept}
\begin{itemize}
  \item The method \code{doCalculation()} is executed once only per iteration and period. Therefore no loops are possible within a single component.
  \item Execution is triggered by the framework: If all wired interfaces (properties of type PacketList starting with \code{in}) have received their information, \code{doCalculation()} is executed. Afterwards, packets to be sent to following components are available \code{in} the \code{out} properties.
  \item Content of \code{in} and \code{out} properties is not shared between different iteration paths and periods. Once the framework has sent outgoing information to following components, the \code{in} and \code{out} property lists are cleared.
\end{itemize}

%NB:h4. 
\subsection*{Implementation Details}
\begin{itemize}
  \item All components are derived from the class\\
        \code{org.pillarone.riskanalytics.core.components.Component}
  \item Use helper methods such as \code{isSenderWired(PacketList sender)} to check if a component will receive information in a specific model or \code{isReceiverWired(PacketList~receiver)} to check if a following component or a collector is wired to an out property.
\end{itemize}

\subsection{Conventions}
\label{subsec:conventions}

Conventions make a developer's life easier by reducing the amount of code to be written and increasing its readability.
In \RA, conventions are used to generate the user interface and database scheme.

\subsection*{Naming convention}
Beside clean code and greater readability, naming conventions increase coding speed
in modern IDEs (integrated development environments) which support code completion.

A component may have several properties. Component property naming employs four prefixes:
\begin{itemize}
  \item \code{parm}: whenever a variable name starts with \code{parm}, it is displayed on the parameterisation user interface automatically. A \code{parm} property may have one of the following types: int, double, String, DateTime, enum.
  \item \code{in}: all \code{in} variables are of type \code{PacketList}. Once they are wired they receive packets from other components.
  \item \code{out}: all out variables are of type \code{PacketList} and can send packets to other components. It is displayed on the result template user interface and on the result page if the property was collected. The collection mode can be specified in the result template.
  \item \code{sub}: a component may have subcomponents. Their name has to start with \code{sub}. They are helpful to provide building blocks and introduce a hierarchy.
\end{itemize}

\note{If a component is written in Java, variables starting with any of these prefixes need a public setter and getter method. In Groovy a variable declared with no access modifier generates a private field with public getter and setter (i.e.~a property). Therefore it is not necessary to write any getter or setter methods with default behaviour in Groovy.}

\subsection*{Grouping of Components and Parameters in the GUI}
The order of parameters, out channels and subcomponents within a component determines the order in
the GUI.

\RA{} comes with a powerful way of deriving a default user interface for all models, components and its
parameters -- a model or component developer does not have to write any GUI code. Yet he can simply group the parameters.

%<!--// todo: add a code snippet and three screenshots: parameter, result template, result page --> 


\subsection{Composed Component}
\label{subsec:composed}

To prevent modelers from always having to start from scratch, the concept of \emph{composed components}
enables a degree of re-use by allowing to define building blocks consisting of several components.
Typical building blocks may be for example lines of business or a reinsurance program.
\todo{A composed component is very similar to a model, except that it can't be executed.}{explain}
The following three sections describe three different scenarios for composed components.
\begin{itemize}
  \item static building blocks containing a \emph{fixed number} of \emph{different} component types
  \item dynamically composed components containing an \emph{arbitrary number} of \emph{equal} components. They are data-driven in the sense that the exact number of subcomponents is specified in the data and not in the model itself. Think of them as containers containing an arbitrary number of predefined subcomponents, wired according to specific rules. Examples: a reinurance program with an arbitrary number of contracts, an arbitrary number of lines of business or claims generators.
  \item multiple calculation phase components. This concept is new and its API not yet stable.
\end{itemize}

\paragraph{Static Building Blocks}
\label{par:statlego}

Whenever several components are always used in the same manner or additional hierarchy levels would facilitate the
understanding of a model, a \code{ComposedComponent} is the appropriate solution. Composed components may be nested.
\todo{Furthermore a ComposedComponent may contain one or many DynamicComposedComponent.}{please explain}
\chapter{Testing Business Logic}
\label{chap:devguide-testing}

\section{Purposes and forms of Testing}
In order to avoid changes breaking existing code different kinds of tests are required to provide an immediate feedback to the developer. According to our code contribution process all tests have to be executed before changes are committed into the source code repository. Furthermore the webapplication hudson will execute all test cases again and provide feedback to the committer if any test has failed.

Two different kinds of tests of business logic which we utilize are:
\begin{itemize}\tightitemize{0pt}
	\item unit tests, for testing single components, strategies, packets or wiring; and
	\item integration tests, to check that model simulations run consistently and 
		completely\footnote{this implicitly includes regression tests, so that old 
		parametrization files will still run in newer \RA{} releases}.
\end{itemize}

\section{Unit Tests}
\begin{itemize}
	\item Directory structure: Each source class should be accompanied by a corresponding unit 
		test class with a parallel classpath \eg, test class 
		
		\code{ClaimsAggregatorTests} in 
		\code{test/unit/org.pillarone.riskanalytics.domain.pc.aggregators} 
		
		tests source class
		
		\code{ClaimsAggregator} in 
		\code{src/groovy/org.pillarone.riskanalytics.domain.pc.aggregators}.
		
	\item Directory structure: A unit test has to be placed in the corresponding package to the source class, \eg \code{ClaimsAggregatorTests} has to be placed in \code{test/unit/org.pillarone.riskanalytics.domain.pc.aggregators} as \code{ClaimsAggregator} is placed in \code{src/groovy/org.pillarone.riskanalytics.domain.pc.aggregators}.
	\item Each unit test class is written in Groovy, has to end with \code{*Tests} in order to be found by the Grails test framework and be derived from \code{GroovyTestCase}.
	\item Test cases are a mean to document the usage of a component, therefore every test class should include a method \code{testUsage()} as documentation of the basic usage of a component.
	\item If business logic calculations is done with doubles or floats, results won't match completly. Therefore it is possible to define an $\epsilon$.
	
	\code{assertEquals 'message', 4, component.testFigure, 1E-8}
	\item If a component contains code throwing exceptions, it's necessary to test if this failures really occure. The syntax is as follows:
	
		\code{shouldFail IllegalArgumentException, \{ component.doCalculation() \}}
		
	The first argument of shouldFail is the expected exception, the second contains a \GroovyClosure{} with the code to be executed in order to get the exception.
	
	TODO: code snippet throwing the same exception type in several blocks: how to make sure exception was thrown where expected? Evaluating exception message?
	\item If objects are used in several test methods it is recommended to create them in static methods. Example from \code{org.pillarone.riskanalytics.domain.pc.reinsurance.contracts}
\begin{lstlisting}[label=lst:getcontract0]
static ReinsuranceContract getContract0() {
    return new ReinsuranceContract(
        parmContractStrategy: ReinsuranceContractStrategyFactory.getContractStrategy(
            ReinsuranceContractType.QUOTASHARE, 
            ["quotaShare": 0.5, 
             "commission": 0.0, 
             "coveredByReinsurer": 1d]))
}
\end{lstlisting}
  This contract can then be used in any other test class. Add parameters to the static functions to get a more flexible usability. Be aware that a new object should be created in every call of the static method to avoid side effects between tests.
  \item When testing several components it won't be sufficient to call \code{doCalculation()} of the first component as this won't trigger following components. Instead \code{start()} has to be called. This method includes publishing of results to following components. Unfortunately it includes the clearing of the out channel lists too. Inspecting out channels for results won't work in this case. Therefore the \code{TestProbe} concept was introduced. It is a probe that can be connected to any out channel and will collect published content, making it available after \code{start()} has been executed. Another scenario for using \code{TestProbe} is to immitate an out channel to be wired in order to trigger calculations. Whenever a \code{TestProbe} is connected to an out channel \code{isSenderWired()} will return \code{true}. Example: \code{def inChannelWired = new TestPretendInChannelWired(claimsGenerator, "inEventSeverities")}
  Examples:
  
  Use \code{def probeGross = new TestProbe(aggregator, "outUnderwritingInfoGross")} to pretend an out channel is wired and
  
  \code{List quotaShareNet = new TestProbe(quotaShare, "outUncoveredClaims").result} to collect outcome from \code{quotaShare.outUncoveredClaims}. Details about the differences of \code{doCalculation()}, \code{execute()} and \code{start()} can be examined in the source code of \code{org.pillarone.riskanalytics.core.components.Component}.
  \item In order to pretend an in channel to be wired \code{TestPretendInChannelWired} has to be used. If an in channel of a component is wired to such a component \code{isReceiverWired()} will be true.
  \item Whenever a \code{ComposedComponent} is tested \code{internalWiring()} has to be called before \code{start()} is executed. Omitting \code{internalWiring()} will result in an execution of the first component within a composed component only.
\end{itemize}
 
\section{Model Tests}

Model tests run a simulation on a specified model and optionally check its output.
The \ixclass{ModelTest} class and associated framework code provide a simple yet 
extensible way to test that simulations run completely (without runtime errors) 
and consistently (reproducibly) across release versions.

Like simulations, model tests take as input a model, a parametrization, and a result 
template. The first element, the model (class name), is required; the latter two, 
parametrization and result template (names) have defaults if they are not given 
explicitly. Additional options specify 
\begin{itemize}\tightitemize{0pt}
	\item how many iterations to run (default: 10),
	\item whether the model test should compare the results collected (default: no),
		\todo{and if so, where the reference data can be found (default naming convention applies)}{verify this statement!}
	\item whether the results should be saved to a file or a database (default: file), and 
	\item the test result's display name and \todo{filename}{or database tablename?}.
\end{itemize}

When collected, the model simulation result consists of tuples of \str{(iteration, 
period, path, field, value)}. When saved to a file, each tuple appears on one line
of a so-called tab-delimited text file with extension \filename{.tsl}.
\todo{}{Where and when is the output directory specified or configured (\eg at run or compile time)?}
The result template defines which values -- \ie, which \str{(path, field)} 
combinations -- to collect.

The driver for each model test is a Groovy class extending \ixclass{ModelTest}.
Each \str{ModelTest} subclass implements a method, \ixmeth{getModelClass}, to tell 
the framework which model (class) it is testing. Likewise, the method 
\ixmeth{shouldCompareResults}, wich returns a \str{boolean} value\footnote{
\str{true} or \str{false}}, tells the framework whether to collect the results
defined in the result template. These are the required \ixclass{ModelTest} elements.

This forces the model testing framework to instantiate the model at runtime, and then run 
a simulation on it using the assigned parameters and result template.
If the option to save the results to a file was selected, the results are written to 
a file in the directory \filename{}.
If the option to compare results was selected, the results are compared with those 
in a file of the same format, which must be given in the directory \filename{}. 
Any differences result in the test failing. If there are no runtime errors and no 
differences between the reference result set and the test's result set, the test passes.

Through this mechanism, one can adopt the following methodology to initially verify 
specific aspects of a model, and subsequently enforce the same behavior, saving
the specification as an artifact \footnote{more precisely, a regression test. These
codify and guarantee application behavior, acting as a measurestick/safeguard/constraint 
to protect against unintended side-effects or incompatible interpretations 
resulting from future code development efforts. This helps not only to fulfill the 
enterprise-level goals of transparency and standards compliance, but also to efficiently
expend development efforts while reaching them.} in \RA.
\begin{enumerate}\tightitemize{0pt}
	\item Generate specific parametrizations and result templates for a model, 
	      targeting specific behavior.
	\item Run the corresponding model test, saving a result file with no comparison.
	\item Inspect and verify the results (once; iterating to this point until correct).
	\item Copy the result file into \filename{\RA{}PC/test/data/}.
	\item Change the test class option to subsequently require comparison,
	      using the copied file from the previous step as the reference result.
\end{enumerate}

The model test class \ixclass{AggregateDrillDownCollectingModeStrategyTests} 
illustrates many of the points mentioned above.

\todo{This section is not yet finished!}{check veracity of and expand/expound on these statements! \eg, AggregateDrillDownCollectingModeStrategyTests}


\chapter{chapter 2 title}
...

\end{document}
\end{lstlisting}
\end{framed}


\section{Our Macros}
\label{sec:ourmac}

Listing \ref{lst:macros} illustrates the use of our macros, which are defined in the file \code{Preamble-Macros.tex}.

\begin{framed}
\begin{lstlisting}[label=lst:macros, caption={Our Macros}, language=TeX]

This is \term{a term} I am introducing.

This is \code{inline code}, such as a Java class name.

This may need changing: \todo{IST}{SOLL}.

\note{This is a note --- that is, one or more full
sentences relating observations or remarks that the
reader should take note of.}

\end{lstlisting}
\end{framed}

\note{\code{{\textbackslash}todo} takes two arguments:
(1) the text to display,
(2) the text that goes in a footnote.}

\note{Use the {\textbackslash}todo macro for notes of things that need to get
done when you want the comment to be visible in the document.}

\caveat{The \code{{\textbackslash}note} and \code{{\textbackslash}caveat} commands should start with
a newline (\code{\textbackslash\textbackslash}) in paragraph mode,
since they may be framed in a box in HTML document formats.}

%\begin{lstlisting}[label=LaTeX~one-liners, backgroundcolor=\color{ListingBackground}]
%
%This (really) is a code block:
%
% \code{inline monospace}
% \term{terms (currently displayed in bold)}
% \emph{empasis (currently displayed in italic style)}
% 
% \chapter{header level 0 (to show up in the TOC, for book documentstyles)}
% 
% \section{header level 1 (to show up in the TOC)}
% 
% \subsection{header level 2 (to show up in the TOC)}
% 
% \subsubsection{header level 3 (to show up in the TOC)}
% 
% \paragraph*{header level 4}
% 
% \subparagraph*{header level 5}
% 
% \note{This is a note (on a new line)}
% 
% \caveat{This is a warning (on a new line)}
%
%\end{lstlisting}

\section{Some Typesetting Rules}
\label{sec:typesetting}

In order to improve the appearance of the documentation we follow some general typsetting rules that are used by all major publishers.

\subsection{Abbreviations containing a dot}
\label{subsec:abbreviationsWithDot}
Abbreviations that contain a dot have to be taken special care of as \LaTeX{} interprets a dot (as well as explanation mark and question mark) always as the end of a sencence and thus add extra whitespace after it (inter-sentence whitespace). This of course is an unwanted behavior in an abbreviation. Dots {\em within} an abbreviation must always be followed by a protected small whitespace of half inter-word length (\code{\textbackslash ,}). Dots {\em at the end} of an abbreviation usually must be followed by an inter-word whitespace. If the abbreviation is followed by a comma, colon or semicolon no space must be added, thus it generally makes sence to define a new command for it in order to manage this behavior.

The command that produces `\ie' is called \code{\textbackslash ie} and is defined as 
\begin{verbatim}
\newcommand{\ie}{i.\,e.\@\xspace}
\end{verbatim}
The \code{\textbackslash xspace} command adds the correct whitespace at the end. Other already implemented abbreviations:
\begin{table}[h]
	\centering
		\begin{tabular}{ll}
 			\code{\textbackslash ie}  & \ie \\
 			\code{\textbackslash eg}  & \eg \\
 			\code{\textbackslash Eg}  & \Eg \\
 			\code{\textbackslash etc} & \etc\\
 			\code{\textbackslash ff}  & \ff
		\end{tabular}
	\caption{Abbreviations containing a dot}
	\label{tab:Abbreviations}
\end{table}

\subsection{Abbreviations following a number}
\label{subsec:abbreviationsAfterNumber}

If a number is followed by an abbreviation \eg `\$'  or `\ff' only a small whitespace (\code{\textbackslash ,}) is used between the number and the abbreviation resulting in `page~100\,\ff' instead of `page~100\ff' or `page~100 \ff'. The correct code is:
\begin{verbatim}
page~100\,\ff
\end{verbatim}

\subsection{Referencing}
\label{subsec:referencing}

When referencing pages, chapters, figures, tables \etc it is important to add a protected inter-word whitespace (\code{\textbackslash\~}) before the reference, otherwise there a linebreak could occur right before the respective number. In a reference the keyword are always written wstarting with a capital letter expect when referencing pages ore lines. Thus the correct refference in the \TeX{} file for this section would be
\begin{verbatim}
see Section~\ref{subsec:abbrebiations}
\end{verbatim}
Note:
\begin{itemize}
\item A reference is {\em never} used without a keyword like `line' or `Section'.\\
			Exception: Equation/Formulae are set in round brackets without a key word.
\item Keywords are never abbreviated: Use `Figure' instead of `Fig.', `page' instead of `p.', \etc
\item Subsections, Subsubsections \etc are refert to as `Section' \eg `see Section~5.2.1'
\end{itemize}


\end{document}
